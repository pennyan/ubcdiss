\chapter{Smtlink Architecture}
\label{ch:smtlink}

\Yan{The introduction to this Chapter should be an expanded version of
the following paragraph. I need to identify which parts require expanding out.}
Smtlink connects the ACL2 interactive theorem prover with an SMT solver using
meta-programming and reflection. The architecture is a pipeline of rule-based
transformers that compiles a term in the untyped first order logic of ACL2 to
the many-sorted first order logic of SMT. Each transformation is guided by
theorems established in ACL2. The final step is a direct translation from ACL2
formulas to their SMT counterparts. Properties over the SMT sorts and their
associated functions are returned back to ACL2 for establishing the soundness of
the last step. This makes the soundness argument straightforward.

\section{Technical Challenges}
\label{sec:challenges}
The many differences between the ACL2 theorem prover and an SMT solver like Z3
proposes many challenges for integrating SMT solvers into the ACL2 theorem
prover. The foremost difference is that the logic of the ACL2 theorem prover is untyped,
while the logic of the SMT solvers is many-sorted. Though an observation is that
ACL2 users tend to have types in mind when defining theorems. This motivated us
to transform untyped ACL2 terms into typed terms.

Another difficulty that comes along with ACL2's untyped logic is that there are
heavy uses of polymorphic functions, for example, \texttt{car}, \texttt{cdr}
and \texttt{cons}. However the many-sorted logic of Z3 requires that functions
or values are associated with a given sort.
For example, in ACL2 one can write \texttt{(cons 1 nil)} where \texttt{cons}
is polymorphic, but in Z3, one needs to write \texttt{IntegerList.cons(1,
  IntegerList.nil)} where \texttt{IntegerList.cons} is a constructor function
for the sort \texttt{IntegerList} and \texttt{IntegerList.nil} is a nullary
constructor for the values of sort \texttt{IntegerList}.
Furthermore, there are commonly used ACL2 features that do not have direct
correspondence in an SMT solver. For example, alist types and its operations.

Last but not least, there exists intricate differences between the two languages.
For example, adding a number with a boolean value in ACL2 is equivalent to
adding a number to the value 0, so \texttt{(+ 1 t)} evaluates to $1$.
However, the Z3 Python bindings in Z3py overload $+$ so that \texttt{1+True}
evaluates to $2$. Another example is when taking \texttt{car} of an empty list.
In ACL2, \texttt{(car nil)} evaluates to \texttt{nil} while in Z3,
\texttt{IntegerList.car(IntegerList.nil)} returns an arbitrary value of sort
\texttt{IntSort()}.

These challenges mean that a naive translation from the ACL2 Lisp language to
the Z3 SMT language can often be wrong and lead to soundness issues. Therefore,
it is crucial to come up with a general solution to address these issues.
\subsection{Untyped vs Many-sorted}
\Yan{This subsection needs to be expanded out.}
Problem: The ACL2 theorem prover is untyped but Z3 is many-sorted.

Solution: Perform type inference over ACL2 terms.

\subsection{Connecting ACL2 types with Z3 types}
\Yan{This subsection needs to be expanded out. State that ACL2 types are more expressive
  than SMT types. Therefore we choose to represent SMT types in ACL2. }
Problem: Through type-inference, we get an ACL2 term that is typed with respect
to ACL2 types. Sometimes, they have direct correspondence with SMT (e.g. Bool
and Integer), sometimes they don't (e.g. (car nil) and association lists).

Solution: Term transformation that transforms a term using ACL2 type (e.g.
alist-p) or functions (e.g. car) to a term using ACL2-embedded SMT types (e.g.
Array-p) or functions (e.g. integer-list-car).

\subsection{Soundness}
\Yan{This subsection needs to be expanded out. Add a footnote that we are
  assuming soundness of the SMT solver and ACL2.}
Problem: How do we know the final transliteration is sound?

Solution: Prove that the ACL2-embedded SMT type satisfy the theory of the
corresponding SMT sorts. 

\section{Meta-programming and Reflection in ACL2}
\label{sec:meta}

Reflection allows us to reason about certain parts of a logical system within
the logical system itself. This generally require a metalogic built into the
logical system. Such a metalogic can be achieved through higher-order logic in a
higher-order logic theorem prover~\cite{harrison1995}.
The ACL2 theorem prover is first-order. In order to provide reflection, it
provides the user with several key features that enables
meta reasoning~\cite{boyer1979, hunt2005, kaufmann2017}.

On the high-level, ACL2 provides clause-processors for user-defined
transformation or simplification of terms.
A correctness theorem must be established to install a verified clause-processor
into the ACL2 system. The correctness theorem ensures that the simplifications
applied to the input clause are sound.
A pseudo-evaluator must be defined for evaluating the functions applied in the
clause-processor. ACL2 also provides reflection for allowing the
clause-processor to consult the logical world for established facts. They can be
used for guiding the transformation or simplification done in a
clause-processor. In addition, hint wrappers and computed-hints are used for
connecting several clause-processors together.

\Yan{Need to define pseudo-term and introduce ACL2 hint}

\subsection{Clause Processors}\label{subsec:cp}
In ACL2, a proof goal is represented as a clause. A clause is a disjunction of
terms. A clause-processor~\cite{clause-processor} takes a proof goal as input
and produces a list of proof goals. Given that the conjunction of the list of
output proof goals are proved, the original proof goal will be established.

For a verified clause-processor, it is ensured that the conjunction of output
proof goals implies the input proof goal. This is done through establishing
the following correctness theorem.

\begin{lstlisting}[style=lisp, label={prog:cp1}, caption={Correctness theorem of
  a user defined clause-proecssor}]
  (defthm correctness-of-a-user-defined-clause-processor-1
    (implies (and (pseudo-term-listp cl)
                  (alistp a)
                  (eval (conjoin-clauses
                          (clauses-result
                            (user-defined-clause-processor cl term state)))
                        a))
             (eval (disjoin cl) a))
    :rule-classes :clause-processor)
\end{lstlisting}

This theorem states that given a list of pseudo-terms \texttt{cl} that
represents the disjunction of terms in the proof goal, an alist \texttt{a} that
represents the environment in which the term will be evaluated with respect to,
if the conjunction of clauses output by the user-defined clause-processor
\texttt{user-defined-clause-processor} evaluated in environment \texttt{a} leads
to true, then the original clause \texttt{cl} evaluated in the same environment
must also lead to true. A proof for the correctness theorem often involves an
induction proof over the structure of the input pseudo-term type.

ACL2 also provides a trusted clause-processor feature. When using a trusted
clause-processor, one doesn't need to prove the above correctness theorem. In
contrast to a verified clause-processor, a trusted clause-processor extends the
ACL2 logical system with the axiom that the trusted clause-processor is sound.
ACL2 asks one to provide a trust tag~\cite{defttag} to acknowledge such an
extension.

\subsection{Pseudo-evaluator}
Since ACL2 is first-order logic, it is not possible to have a complete evaluator
because a complete evaluator uses second-order logic. In particular,
one cannot write an evaluator \texttt{eval} that can evaluator a function
\texttt{f} that is going to be defined. However, we can imagine writing a
constrained function that can evaluate terms using a subset of defined
functions. This evaluator treats each function as a single case and evaluates
them separately. In addition, it checks if such a function already exist. This
is called a pseudo-evaluator. 

Note that the correctness theorem shown in Section~\ref{subsec:cp} uses a
pseudo-evaluator called \texttt{eval}. A pseudo-evaluator is a constrained
function, with constraints (expressed as ACL2 theorems) for the evaluation of
variables, quoted constants, lambda-expressions, and a specified set of
functions.  The value produced by the pseudo-evaluator for the evaluation of any
function outside of that set is unconstrained.  The actual evaluation of an ACL2
term satisfies all of the constraints of the pseudo-evaluator. Therefore, if one
can establish that the correctness theorem holds using the pseudo-evaluator,
then one can infer that the correctness theorem holds for a full evaluator,
since all constraint theorems proved over a pseudo-evaluator are true theorems
in a full evaluator as well~\cite{kaufmann2017}. To see this, take the following
pseudo-evaluator as an example:

\begin{lstlisting}[style=lisp]
  (defevaluator demo-ev demo-ev-list
    ((if x y z) (equal x y)) :namedp t)
\end{lstlisting}

The above form defines an evaluator called \texttt{demo-ev}. The
\texttt{demo-ev} and \texttt{demo-ev-list} form a mutual recursion of evaluators
that work on a \texttt{pseudo-termp} and a \texttt{pseudo-term-listp}
respectively. In this pseudo-evaluator, we specify that it should know about the
evaluation constraints for \texttt{if} and \texttt{equal}. Together, ten
constraints are proved.
The first eight constraints are common constraints that are defined for all
pseudo-terms. They define the evaluation rules for evaluating a pseudo-term. The
last two rules, \texttt{demo-ev-of-if-call} and \texttt{demo-ev-of-equal-call}
are evaluation rules for \texttt{if} and \texttt{equal}.

\begin{lstlisting}[style=lisp]
(defthmd demo-ev-of-fncall-args
  (implies (and (consp x)
                (syntaxp (not (equal a ''nil)))
                (not (equal (car x) 'quote)))
           (equal (demo-ev x a)
                  (demo-ev (cons (car x)
                                 (kwote-lst (demo-ev-list (cdr x) a)))
                           nil))))
(defthm demo-ev-of-variable
  (implies (symbolp x)
           (equal (demo-ev x a)
                  (and x (cdr (assoc-equal x a))))))
(defthm demo-ev-of-quote
  (implies (and (consp x) (equal (car x) 'quote))
           (equal (demo-ev x a) (cadr x))))
(defthm demo-ev-of-lambda
  (implies (and (consp x) (consp (car x)))
           (equal (demo-ev x a)
                  (demo-ev (caddr (car x))
                           (pairlis$ (cadr (car x))
                                     (demo-ev-list (cdr x) a))))))
(defthm demo-ev-list-of-atom
  (implies (not (consp x-lst))
           (equal (demo-ev-list x-lst a) nil)))
(defthm demo-ev-list-of-cons
  (implies (consp x-lst)
           (equal (demo-ev-list x-lst a)
                  (cons (demo-ev (car x-lst) a)
                        (demo-ev-list (cdr x-lst) a)))))
(defthmd demo-ev-of-nonsymbol-atom
  (implies (and (not (consp x)) (not (symbolp x)))
           (equal (demo-ev x a) nil)))
(defthmd demo-ev-of-bad-fncall
  (implies (and (consp x)
                (not (consp (car x)))
                (not (symbolp (car x))))
           (equal (demo-ev x a) nil)))
(defthm demo-ev-of-if-call
  (implies (and (consp x) (equal (car x) 'if))
           (equal (demo-ev x a)
                  (if (demo-ev (cadr x) a)
                      (demo-ev (caddr x) a)
                    (demo-ev (cadddr x) a)))))
(defthm demo-ev-of-equal-call
  (implies (and (consp x) (equal (car x) 'equal))
           (equal (demo-ev x a)
                  (equal (demo-ev (cadr x) a)
                         (demo-ev (caddr x) a)))))
\end{lstlisting}

\texttt{demo-ev-of-if-call} says that if \texttt{x} is a pseudo-term of an
\texttt{if} call, evaluating \texttt{x} with respect to the environment
\texttt{a} is equivalent to evaluating the ``condition'', ``then'' and ``else''
branches seperately and then form an ``if'' after the evaluations.
Similarly for \texttt{demo-ev-of-equal-call}, evaluating an ``equal'' term is
the same as evaluating the two sides and then test for equality.
The above constraints are the only facts ACL2 knows about \texttt{demo-ev} and
\texttt{demo-ev-list}. When evaluating a transformation in a clause-processor,
if the transformation goes beyond the set of constraints, one cannot prove
soundness of the reflection. On the other hand, if these constraints are enough
for establishing soundness of a clause-processor, then we know the soundness of
such a clause-processor also holds in a full evaluator.

Note that \texttt{demo-ev-of-if-call} and \texttt{demo-ev-of-equal-call} don't
have constraints over the wellformedness of the quoted term. So, essentially,
one can verify a clause-processor that returns a malformed term using
pseudo-evaluators. For example, a verified clause-processor can return an ``if''
term with an arbitrary fourth argument.
However, the ACL2 waterfall ensures that any produced subgoals must be
well-formed. For example, every function symbol must exist in the logical world
and its number of arguments must also be correct~\cite{kaufmann2015, termp}.

\subsection{Meta-extract}
When defining a pseudo-evaluator, we define how functions are evaluated to
validate the transformations done in a clause-processor. Sometimes we want to
know more than function evaluation.

For example, defining a function in ACL2 generally creates a :definition
rule\footnote{ACL2 allows for constrained functions (e.g. demo-ev described
  above) and other logical function definitions that don’t produce :definition
  rules, but these are not supported in the current version of Smtlink.} that
equates a call to the function to an instance of the function body.
One can imagine using this rule in a clause-processor to replace a function call
with its definition.
Further from it, another example would be to simplify the proof goal using
other kind of rewrite rules. In addition, useful type information can also be
found from the \texttt{type-set} facts.

To achieve this goal, ACL2 provides meta-extract~\cite{meta-extract} that allows
a clause-processor or meta-function to extract established facts from the ACL2
logical world. In addition to that, it provides facilities for proving soundness
of the extraction. This is the key feature of computational reflection that is
heavily used in Smtlink.

Meta-extract can be defined using the following form (Note that
\texttt{def-meta-extract} requires that a minimal set of functions must be
defined in the evaluator, hence we define a new version of \texttt{demo-ev}):
\begin{lstlisting}[style=lisp]
  (defevaluator demo-ev demo-ev-list
    ((typespec-check ts x)
     (if a b c)
     (equal a b)
     (not a)
     (iff a b)
     (implies a b)))
  (def-meta-extract demo-ev demo-ev-list)
\end{lstlisting}
This above form produces two meta-extract hypotheses that can be used in the
soundness theorem of a clause-processor, one for extracting contextual facts and
the other for extracting global facts --
\texttt{(demo-ev-meta-extract-global-facts)}. This thesis focuses on the support
for extracting global facts.
A key theorem is proved for meta-extract of global facts:
\begin{lstlisting}[style=lisp]
  (defthm demo-ev-meta-extract-formula
    (implies (and (demo-ev-meta-extract-global-facts)
                  (equal (w st) (w state)))
             (demo-ev (meta-extract-formula name st)
                      a)))
\end{lstlisting}
The hypothesis \texttt{(demo-ev-meta-extract-global-facts)} states that global
facts extracted by meta-extract is always valid, i.e. there doesn't exist an
object for which when its formula is extracted from the logical world, results
in an invalid formula, evaluated in any environment. If the hypothesis holds,
then we know we can use \texttt{meta-extract-formula} to extract a rule with a
given name \texttt{name} from the logical world saved in the current
\texttt{state}, and the evaluation of that formula is true in any environment.

Given that we have defined the facilities of meta-extract for our evaluator, we
can use \texttt{meta-extract-formula} in our clause-processor for extracting
established facts. For a verified clause-processor, we prove the following
soundness theorem in the end:
\begin{lstlisting}[style=lisp]
  (defthm correctness-of-a-user-defined-clause-processor-2
    (implies (and (demo-ev-meta-extract-global-facts)
                  (pseudo-term-listp cl)
                  (alistp a)
                  (eval (conjoin-clauses
                          (clauses-result
                            (user-defined-clause-processor
                              cl term state)))
                        a))
             (eval (disjoin cl) a))
  :rule-classes :clause-processor)
\end{lstlisting}
In comparison to theorem~\ref{prog:cp1}, one additional hypothesis
\texttt{demo-ev-meta-extract-global-facts} is added. This hypothesis allows us
to assume that formulas extracted by using meta-extract are valid formulas. By
using meta-extract, we strengthen the ACL2 logical system by trusting
meta-extract. Note that it is possible to remove the trust for meta-extract by
returning each of the extracted formulas as a subgoal. However, doing so would
complicate the Smtlink code and negatively impact the performance.

\subsection{Computed hints}
A hint-wrapper~\cite{hint-wrapper} is a way of supplying hints in the statement
of a theorem. For example, we can have,

\begin{lstlisting}[style=lisp]
  (defund foo (x y) (or (implies x y) (and x (not y))))
  (thm
    (implies (hint-wrapper '(:in-theory (enable foo)))
             (foo x y)))
\end{lstlisting}

The \texttt{hint-wrapper} function always returns \texttt{'t}, therefore the
theorem statement is equivalent to \texttt{(foo x y)}.
That it, the hint-wrapper term doesn't change the logical meaning of the theorem 
statement.
However, it provides hints to be used for a computed-hint. A computed-hint is a
program that monitors every subgoal produced during a proof. It recognizes
patterns of a hint-wrapper and installs hints provided by the hint-wrapper into
subsequent goals. In this example, a computed-hint will install the hint
\texttt{:in-theory (enable foo)} into the next subgoal, allowing the theorem to
be proved.

Note that a computed-hint is also a form of proof automation. It allows hints to
be applied intelligently without manual user intervention. However, it is
different from reflection, in that it merely guides the proof and does not
change the theorem statement. Therefore, there isn't any soundness concern with
uses of computed-hints.

\section{The Architecture}
\label{sec:arch}

\Yan{This section is a high-level description of the architecture. There should be a
graph. This section doesn't talk about each step in the pipeline. Instead it
desribes how the clause processors are connected together through computed hints
and the connection to Z3.}

Figure~\ref{fig:arch} shows the Smtlink reflection-based architecture composing
of a chain of verified clause-processors leading to a trusted clause-processor.
The verified steps include user hint processing, user hypotheses introduction,
function expansion, type inference, term replacement, trusted clause-processor
hint generation, and type declaration extraction.
The user can choose between two trusted clause-processors that transliterate
from an ACL2 proof goal to an SMT goal: trusted clause-processor and customized
trusted clause-processor.
The customized trusted clause-processor gives the user the freedom to program
their own solver class using the Z3py interface.
Note that the architecture of Smtlink is chosen to be a linear chain of steps
for simplicity. Smtlink is highly customizable and one can imagine changing the
architecture to introduce branches, and even loops. However, these extensions
are beyond the topic of this thesis.

Each clause-processor serves a certain purpose in the transformation from a
top-level ACL2 proof goal to an SMT term.
A common theme of the chain of verified clause-processors is that they make
heavy use of rule-based reflection. The user can provide rules (established
facts in the ACL2 world) through the Smtlink hint to guide the reflection
transformations achieved in each verified clause processor.

\begin{figure}\begin{center}
  \resizebox{1.0\textwidth}{!}{\chapter{The Reflection-based Integration Architecture}
\label{ch:arch}

% There doesn't exist a direct correspondence of the alist type in Z3. One of the
% idea is to use \texttt{ArraySort} to represent alists. However, there are
% several difficulties.
% First, ACL2 alists can be empty but Z3 arrays are infinite. We solve this
% problem by using constant arrays initialized with \texttt{nil} constants to
% represent empty alists.
% Second, the destructor function \texttt{assoc-equal} doesn't correspond directly
% to Z3's \texttt{Select} function. More specifically, \texttt{assoc-equal}
% produces a cons-pair, while \texttt{Select} produces a value.
% To solve these problems, for each alist type, we provide a corresponding array
% definition in ACL2.
% As is described in Chapter~\ref{ch:arch}, the replacement clause-processor
% replaces uses of alist recognizers, empty alists, \texttt{acons},
% \texttt{assoc-equal} and \texttt{cdr} with their corresponding array functions.

}
  \caption{The Smtlink Architecture}\label{fig:arch}
\end{center}\end{figure}

\begin{itemize}
\item User hint processing: The first verified
  clause-processor does user hint processing to process the smtlink hints.
  \item User hypotheses intro: For any nontrivial theorem, it is almost always a requirement to provide lemmas
    that can help prove the target proof goal. Typically, the user might have proven
    a bunch of useful theorems that can help the SMT solver to discharge the proof
    goal. Sometimes, providing useful theorems improves the performance of SMT
    solving, in other times, it is necessary to know such fact in order to prove the
    original goal. Simply providing all lemmas to the SMT solver is a risk for
    exponential blowup. The user hypotheses introduction clause-processor allows the
    user to introduce only ``interesting'' lemmas to help proving a proof goal.

  \item Function expansion: Often times, users of theorem provers will define functions for proper
    abstraction of a nontrivial system. Some of these functions might be recursive.
    It would be rather tedious to make the user manually expands proof goals into
    SMT primitives. The function expansion clause-processor automatically expands
    out functions in a way that is guided by user hints.

  \item Type inference: To bridge the untyped logic of ACL2 with the many-sorted logic of SMT, Smtlink
    does type inference. The type inference clause-processor has two purposes.
    First, it assigns a type to each subterm of the proof goal. These types of
    subterms ensure that a consistent set of types can be assigned to the ACL2 term.
    Second, the type information is passed onto later stages like term replacement
    and the final trusted clause-processor for further transformation of the proof
    goal.

  \item Term replacement: Further from the problem of the different type system of ACL2 and SMT, functions
    in ACL2 are polymorphic whereas functions in Z3 are typed. For example, in ACL2,
    \texttt{cons} can be applied to any list, however, in Z3,
    \texttt{IntegerList.cons} is the function to apply to an \texttt{IntegerList}.
    Smtlink will need to identify types of subterms and replace them with the
    appropriately typed versions. For example, we define in ACL2 a function
    \texttt{integer-list->cons} and replace \texttt{(cons x y)} with
    \texttt{(integer-list->cons x y)}. The term replacement clause-processor also
    handles the case when certain sort of SMT doesn't exist in ACL2, which will be
    discussed in later sections.

  \item Trusted clause-processor hint generation: The trusted clause-processor hint generation clause-processor generates hints
    that are used by the trusted clause-processor.
  \item Type declaration extraction: The last verified
    clause-processor does type declaration extraction to extract type predicates
    from the disjunct of clauses. At this point, the proof goal is annotated with
    necessary hypotheses, fully expanded out, well-typed, and replaced with SMT
    corresponding functions.
  \item The trusted clause-processors: Smtlink then runs the trusted clause-processor to
    translate the flattened goal into SMT and invoke the SMT solver.
\end{itemize}

\Yan{Show an example that run through the architecture.}

\section{The Verified Clause-processors}
\label{sec:vcp}
\Yan{A top-level description of how the rule-guided transformations work.}

\subsection{User Hypotheses Introduction}
The user hypotheses introduction clause-processor allows the user to add
additional facts to assist the SMT solvers.

The hypotheses are theorems already established in ACL2. Their names and
instantiation information are provided to Smtlink through the hint interface.

Given a rule, Smtlink uses meta-extract to extract the body of the rule,
substitute the rule properly and add the negation of it into the disjunct of
clauses of the goal.

The user can attach any hypotheses onto the goal soundly this way.

\subsection{Function Expansion}
The function expansion clause-processor flattens the proof goal through
beta-reduction. Non-recursive function are expanded out by default. Recursive
functions can be expanded to a user-provided level. The last recursive call is
treated as an uninterpreted function. It is also possible to treat non-recursive
function as uninterpreted function by supplying an expansion level of 0.

Given an input goal, the function expansion clause-processor recursively
traverses the goal. When expanding out a function definition, it uses
meta-extract to extract the equivalence rule for the function definition.
It unifies the input argument with the formals of the function to get a
substitution. Then the clause-processor does beta-reduction to expand the
function definition. Lambdas are fully expanded out in this process as well.

Through meta-extract, we can verify in ACL2 that this expansion process is
sound. The hint interface allows the user to have flexibility in how they want
to control the expansion. 

\subsection{Type Inference}
Smtlink has a type inference engine fully guided by existing typing rules in
ACL2. Since any ACL2 function could be a typing rule, this allows a wide range
of types to be inferred. The user can choose to enable difference set of rules
and the inferred types will be different.

The type inference engine is composed of two steps. First, a bottom-up pass is
employed to get all possible types of all subterms. We store the information in
a data structure called typed-term.

Second, a top-down pass will choose from all possible types a set of types that
are consistent for assisting the translation to SMT.

\subsection{Term Transformation (Conditional rewriting)}
Smtlink allows term transformation for rewriting the goal. The motivation of
term transformation is battle the gap between ACL2 and the SMT solvers. But it
could also be used as a rewriting capability.

The user provides a set of conditioned equality rules. Smtlink uses meta-extract
to soundly get the rule. If the condition of the rule could be satisfied, it
will apply the rule and rewrite the subterm to the rhs of the rule. At the same
time, it updates the type inference information.

\Yan{Mention alists to arrays}

\section{The Trusted Clause-processors}
\label{sec:tcp}
\Yan{Describe what the shape the input to a trusted clause-processor looks like}
\Yan{Describe on a high-level what the trusted clause-processor does}

\subsection{SMT Sorts}
BoolSort, IntSort, RealSort, Datatypes (algebraic data type), Abstract types, Arrays
\Yan{List the SMT sorts supported and the functions associated with them}

\Yan{List the properties verified with each sort}

\Yan{Somewhere I have to mention the treating of equal functions. Should it be here?}

\subsection{Uninterpreted Functions}
User-defined functions left in the goal are translated as uninterpreted
functions. The type signature of the uninterpreted functions are returned as
subgoals back to ACL2 for verification.

\section{Summary}
\label{sec:summary}

\begin{comment}

\chapter{The Reflection-based Integration Architecture}
\label{ch:arch}

The many differences between the ACL2 theorem prover and an SMT solver like Z3
proposes many challenges for integrating SMT solvers into the ACL2 theorem
prover.

The foremost difference is that the logic of the ACL2 theorem prover is untyped,
while the logic of the SMT solvers is many-sorted. First, we identify a subset
of the sorts of the SMT solvers that are embedded in the ACL2 logic. A proof
goal in ACL2 needs to be a well-typed term with respect to this subset of sorts,
so that it can be transliterated into its SMT correspondence.

Another difficulty that comes along with ACL2's untyped logic is that there are
heavy uses of polymorphic functions, for example, \texttt{car}, \texttt{cdr}
and \texttt{cons}. However the many-sorted logic of Z3 requires that functions
or values are associated with a given sort.
For example, in ACL2 one can write \texttt{(cons 1 nil)} where \texttt{cons}
is polymorphic, but in Z3, one needs to write \texttt{IntegerList.cons(1,
  IntegerList.nil)} where \texttt{IntegerList.cons} is a constructor function
for the sort \texttt{IntegerList} and \texttt{IntegerList.nil} is a nullary
constructor for the values of sort \texttt{IntegerList}.
Furthermore, there are commonly used ACL2 features that do not have direct
correspondence in an SMT solver. For example, alist types and its operations.

Last but not least, there exists intricate differences between them.
These intricate differences, if not handled correctly, can lead to severe
soundness issue in the integration.
For example, adding a number with a boolean value in ACL2 is equivalent to
adding a number to the value 0, so \texttt{(+ 1 t)} evaluates to $1$.
However, the Z3 Python bindings in Z3py overload $+$ so that \texttt{1+True}
evaluates to $2$. Another example is when taking \texttt{car} of an empty list.
In ACL2, \texttt{(car nil)} evaluates to \texttt{nil} while in Z3,
\texttt{IntegerList.car(IntegerList.nil)} returns an arbitrary value of sort
\texttt{IntSort()}.

These challenges mean that a naive translation from the ACL2 Lisp language to
the Z3 SMT language can often be wrong and lead to soundness issues. Therefore,
it is crucial to come up with a general solution to address these issues.
This chapter discusses my solution of a reflection-based integration
architecture called Smtlink. This architecture takes a proof goal in ACL2,
applies several sound transformation steps to generate a goal that is
well-typed under the type system of the SMT solver and preserves the logical
meaning of the original proof goal.
A final transliteration step is then applied to generate an SMT term. This
architecture takes heavy use of the metaprogramming and reflection techniques.

Metaprogramming is based on the observation that programs are data: programs can
be represented in forms that can be stored and manipulated by computers.
In metaprogramming, the programmer can write a function that takes a program, or
a program fragment as an argument, and returns one or more programs or fragments
as a result. LISP macros are an example of metaprogramming.

Reflection is a technique that bears many meanings in the many fields of
mathematics and computer science.
In this thesis, reflection refers to the  ACL2 theorem prover's support for
reasoning about both a quoted term and the value produced by its evaluation.
For example, given a quoted term, \texttt{quoted-term-1}, a function in Smtlink
could derive a new quoted term \texttt{quoted-term-2}, and use the reflection
capabilities of ACL2 to show that,
\begin{lstlisting}[style=lisp]
  (implies (eval quoted-term-2 env) (eval quoted-term-2 env))
\end{lstlisting}
It says that the evaluation of \texttt{quoted-term-2} implies the evaluation of
\\
\texttt{quoted-term-1} in any environment \texttt{env}.
This soundness theorem establishes that theorems derived from the application
of reflection are a sound addition to the given logical system.
This kind of reflection is referred to as computational reflection
in~\cite{harrison1995}.
Computational reflection alleviates the tedium of repeatedly applying the same
set of proof techniques for similar proofs.

For Smtlink, we develop a rule-guided reflection-based architecture for sound
transformation from an ACL2 proof goal into a SMT proof goal embedded in ACL2. 
The architecture is a pipeline of rule-guided reflection steps.
These reflection steps include but are not limited to, function expansion, type
inference and term replacement.
Each reflection-based step takes a proof goal that's the output of a previous
reflection, inspect and transform the input proof goal and output the
resulting proof goal to another reflection-based step.
The transformations are rule-guided. This means that the user supply rules for
guiding the transformations. For example, the type inference step takes rules
that specify function return types and use them for guiding the type inference
of subterms of function calls.

This way, we modularize the architecture into several well-contained
transformations.
The reflection-based Smtlink architecture has several advantages. First, each
transformation step is verified to be sound reflections within ACL2, except for
the last step that translates directly from an ACL2 proof goal into an SMT term.
Second, the architecture is modularized and therefore is easily extensible. An
intrepid user might want to extend the Smtlink architecture for their own use,
and all they have to do is to add a step into the pipeline of reflection-based
transformations. Soundness can be easily achieved in these user extensions.
Third, the transformations are customizable. The user has considerable freedom
to customize how they want the proof goal to be discharged by supplying rules.
These rules are extracted from the ACL2 logical world using a mechanism that
allows sound customized transformations to be done.

This chapter presents the reflection-based architecture of Smtlink.
Section~\ref{sec:background} introduces several key concepts that enable the
computational reflection in ACL2.
Section~\ref{sec:arch} provides a high-level picture of the architecture, and
presents each steps in the reflection-based architecture.
Section~\ref{sec:summary} summarizes this chapter.

\section{Background}
\label{sec:background}
Reflection allows us to reason about certain parts of a logical system within
the logical system itself. This generally require a metalogic built into the
logical system. Such a metalogic can be achieved through higher-order logic in a
higher-order logic theorem prover~\cite{harrison1995}.
The ACL2 theorem prover is first-order. In order to provide reflection, it
provides the user with several key features that enables
meta reasoning~\cite{boyer1979, hunt2005, kaufmann2017}.

On the high-level, ACL2 provides clause-processors for user-defined
transformation or simplification of terms.
A correctness theorem must be established to install a verified clause-processor
into the ACL2 system. The correctness theorem ensures that the simplifications
applied to the input clause are sound.
A pseudo-evaluator must be defined for evaluating the functions applied in the
clause-processor. ACL2 also provides reflection for allowing the
clause-processor to consult the logical world for established facts. They can be
used for guiding the transformation or simplification done in a
clause-processor. In addition, hint wrappers and computed-hints are used for
connecting several clause-processors together.

In the rest of this section, each of these features will be discussed. Together,
they enable an efficient mechanism for using reflection to automate the proof
process.

\subsection{Clause Processors}\label{subsec:cp}
In ACL2, a proof goal is represented as a clause. A clause is a disjunction of
terms. A clause-processor~\cite{clause-processor} takes a proof goal as input
and produces a list of proof goals. Given that the conjunction of the list of
output proof goals are proved, the original proof goal will be established.

For a verified clause-processor, it is ensured that the conjunction of output
proof goals implies the input proof goal. This is done through establishing
the following correctness theorem.

\begin{lstlisting}[style=lisp, label={prog:cp1}, caption={Correctness theorem of
  a user defined clause-proecssor}]
  (defthm correctness-of-a-user-defined-clause-processor-1
    (implies (and (pseudo-term-listp cl)
                  (alistp a)
                  (eval (conjoin-clauses
                          (clauses-result
                            (user-defined-clause-processor cl term state)))
                        a))
             (eval (disjoin cl) a))
    :rule-classes :clause-processor)
\end{lstlisting}

This theorem states that given a list of pseudo-terms \texttt{cl} that
represents the disjunction of terms in the proof goal, an alist \texttt{a} that
represents the environment in which the term will be evaluated with respect to,
if the conjunction of clauses output by the user-defined clause-processor
\texttt{user-defined-clause-processor} evaluated in environment \texttt{a} leads
to true, then the original clause \texttt{cl} evaluated in the same environment
must also lead to true. A proof for the correctness theorem often involves an
induction proof over the structure of the input pseudo-term type.

ACL2 also provides a trusted clause-processor feature. When using a trusted
clause-processor, one doesn't need to prove the above correctness theorem. In
contrast to a verified clause-processor, a trusted clause-processor extends the
ACL2 logical system with the axiom that the trusted clause-processor is sound.
ACL2 asks one to provide a trust tag~\cite{defttag} to acknowledge such an
extension.

\subsubsection{Pseudo-evaluator}
Since ACL2 is first-order logic, it is not possible to have a complete evaluator
because a complete evaluator apparently uses second-order logic. In particular,
one cannot write an evaluator \texttt{eval} that can evaluator a function
\texttt{f} that is going to be defined. However, we can imagine writing a
constrained function that can evaluate terms using a subset of defined
functions. This evaluator treats each function as a single case and evaluates
them separately. In addition, it checks if such a function already exist. This
is called a pseudo-evaluator. 

Note that the correctness theorem shown in Section~\ref{subsec:cp} uses a
pseudo-evaluator called \texttt{eval}. A pseudo-evaluator is a constrained
function, with constraints (expressed as ACL2 theorems) for the evaluation of
variables, quoted constants, lambda-expressions, and a specified set of
functions.  The value produced by the pseudo-evaluator for the evaluation of any
function outside of that set is unconstrained.  The actual evaluation of an ACL2
term satisfies all of the constraints of the pseudo-evaluator. Therefore, if one
can establish that the correctness theorem holds using the pseudo-evaluator,
then one can infer that the correctness theorem holds for a full evaluator,
since all constraint theorems proved over a pseudo-evaluator are true theorems
in a full evaluator as well~\cite{kaufmann2017}. To see this, take the following
pseudo-evaluator as an example:

\begin{lstlisting}[style=lisp]
  (defevaluator demo-ev demo-ev-list
    ((if x y z) (equal x y)) :namedp t)
\end{lstlisting}

The above form defines an evaluator called \texttt{demo-ev}. The
\texttt{demo-ev} and \texttt{demo-ev-list} form a mutual recursion of evaluators
that work on a \texttt{pseudo-termp} and a \texttt{pseudo-term-listp}
respectively. In this pseudo-evaluator, we specify that it should know about the
evaluation constraints for \texttt{if} and \texttt{equal}. Together, ten
constraints are proved.
The first eight constraints are common constraints that are defined for all
pseudo-terms. They define the evaluation rules for evaluating a pseudo-term. The
last two rules, \texttt{demo-ev-of-if-call} and \texttt{demo-ev-of-equal-call}
are evaluation rules for \texttt{if} and \texttt{equal}.

\begin{lstlisting}[style=lisp]
(defthmd demo-ev-of-fncall-args
  (implies (and (consp x)
                (syntaxp (not (equal a ''nil)))
                (not (equal (car x) 'quote)))
           (equal (demo-ev x a)
                  (demo-ev (cons (car x)
                                 (kwote-lst (demo-ev-list (cdr x) a)))
                           nil))))
(defthm demo-ev-of-variable
  (implies (symbolp x)
           (equal (demo-ev x a)
                  (and x (cdr (assoc-equal x a))))))
(defthm demo-ev-of-quote
  (implies (and (consp x) (equal (car x) 'quote))
           (equal (demo-ev x a) (cadr x))))
(defthm demo-ev-of-lambda
  (implies (and (consp x) (consp (car x)))
           (equal (demo-ev x a)
                  (demo-ev (caddr (car x))
                           (pairlis$ (cadr (car x))
                                     (demo-ev-list (cdr x) a))))))
(defthm demo-ev-list-of-atom
  (implies (not (consp x-lst))
           (equal (demo-ev-list x-lst a) nil)))
(defthm demo-ev-list-of-cons
  (implies (consp x-lst)
           (equal (demo-ev-list x-lst a)
                  (cons (demo-ev (car x-lst) a)
                        (demo-ev-list (cdr x-lst) a)))))
(defthmd demo-ev-of-nonsymbol-atom
  (implies (and (not (consp x)) (not (symbolp x)))
           (equal (demo-ev x a) nil)))
(defthmd demo-ev-of-bad-fncall
  (implies (and (consp x)
                (not (consp (car x)))
                (not (symbolp (car x))))
           (equal (demo-ev x a) nil)))
(defthm demo-ev-of-if-call
  (implies (and (consp x) (equal (car x) 'if))
           (equal (demo-ev x a)
                  (if (demo-ev (cadr x) a)
                      (demo-ev (caddr x) a)
                    (demo-ev (cadddr x) a)))))
(defthm demo-ev-of-equal-call
  (implies (and (consp x) (equal (car x) 'equal))
           (equal (demo-ev x a)
                  (equal (demo-ev (cadr x) a)
                         (demo-ev (caddr x) a)))))
\end{lstlisting}

\texttt{demo-ev-of-if-call} says that if \texttt{x} is a pseudo-term of an
\texttt{if} call, evaluating \texttt{x} with respect to the environment
\texttt{a} is equivalent to evaluating the ``condition'', ``then'' and ``else''
branches seperately and then form an ``if'' after the evaluations.
Similarly for \texttt{demo-ev-of-equal-call}, evaluating an ``equal'' term is
the same as evaluating the two sides and then test for equality.
The above constraints are the only facts ACL2 knows about \texttt{demo-ev} and
\texttt{demo-ev-list}. When evaluating a transformation in a clause-processor,
if the transformation goes beyond the set of constraints, one cannot prove
soundness of the reflection. On the other hand, if these constraints are enough
for establishing soundness of a clause-processor, then we know the soundness of
such a clause-processor also holds in a full evaluator.

Note that \texttt{demo-ev-of-if-call} and \texttt{demo-ev-of-equal-call} don't
have constraints over the wellformedness of the quoted term. So, essentially,
one can verify a clause-processor that returns a malformed term using
pseudo-evaluators. For example, a verified clause-processor can return an ``if''
term with an arbitrary fourth argument.
However, the ACL2 waterfall ensures that any produced subgoals must be
well-formed. For example, every function symbol must exist in the logical world
and its number of arguments must also be correct~\cite{kaufmann2015, termp}.

\subsubsection{Meta-extract}
When defining a pseudo-evaluator, we define how functions are evaluated to
validate the transformations done in a clause-processor. Sometimes we want to
know more than function evaluation.

For example, defining a function in ACL2 generally creates a :definition
rule\footnote{ACL2 allows for constrained functions (e.g. demo-ev described
  above) and other logical function definitions that don’t produce :definition
  rules, but these are not supported in the current version of Smtlink.} that
equates a call to the function to an instance of the function body.
One can imagine using this rule in a clause-processor to replace a function call
with its definition.
Further from it, another example would be to simplify the proof goal using
other kind of rewrite rules. In addition, useful type information can also be
found from the \texttt{type-set} facts.

To achieve this goal, ACL2 provides meta-extract~\cite{meta-extract} that allows
a clause-processor or meta-function to extract established facts from the ACL2
logical world. In addition to that, it provides facilities for proving soundness
of the extraction. This is the key feature of computational reflection that is
heavily used in Smtlink.

Meta-extract can be defined using the following form (Note that
\texttt{def-meta-extract} requires that a minimal set of functions must be
defined in the evaluator, hence we define a new version of \texttt{demo-ev}):
\begin{lstlisting}[style=lisp]
  (defevaluator demo-ev demo-ev-list
    ((typespec-check ts x)
     (if a b c)
     (equal a b)
     (not a)
     (iff a b)
     (implies a b)))
  (def-meta-extract demo-ev demo-ev-list)
\end{lstlisting}
This above form produces two meta-extract hypotheses that can be used in the
soundness theorem of a clause-processor, one for extracting contextual facts and
the other for extracting global facts --
\texttt{(demo-ev-meta-extract-global-facts)}. This thesis focuses on the support
for extracting global facts.
A key theorem is proved for meta-extract of global facts:
\begin{lstlisting}[style=lisp]
  (defthm demo-ev-meta-extract-formula
    (implies (and (demo-ev-meta-extract-global-facts)
                  (equal (w st) (w state)))
             (demo-ev (meta-extract-formula name st)
                      a)))
\end{lstlisting}
The hypothesis \texttt{(demo-ev-meta-extract-global-facts)} states that global
facts extracted by meta-extract is always valid, i.e. there doesn't exist an
object for which when its formula is extracted from the logical world, results
in an invalid formula, evaluated in any environment. If the hypothesis holds,
then we know we can use \texttt{meta-extract-formula} to extract a rule with a
given name \texttt{name} from the logical world saved in the current
\texttt{state}, and the evaluation of that formula is true in any environment.

Given that we have defined the facilities of meta-extract for our evaluator, we
can use \texttt{meta-extract-formula} in our clause-processor for extracting
established facts. For a verified clause-processor, we prove the following
soundness theorem in the end:
\begin{lstlisting}[style=lisp]
  (defthm correctness-of-a-user-defined-clause-processor-2
    (implies (and (demo-ev-meta-extract-global-facts)
                  (pseudo-term-listp cl)
                  (alistp a)
                  (eval (conjoin-clauses
                          (clauses-result
                            (user-defined-clause-processor
                              cl term state)))
                        a))
             (eval (disjoin cl) a))
  :rule-classes :clause-processor)
\end{lstlisting}
In comparison to theorem~\ref{prog:cp1}, one additional hypothesis
\texttt{demo-ev-meta-extract-global-facts} is added. This hypothesis allows us
to assume that formulas extracted by using meta-extract are valid formulas. By
using meta-extract, we strengthen the ACL2 logical system by trusting
meta-extract. Note that it is possible to remove the trust for meta-extract by
returning each of the extracted formulas as a subgoal. However, doing so would
complicate the Smtlink code and negatively impact the performance.

\subsection{Computed hints}
A hint-wrapper~\cite{hint-wrapper} is a way of supplying hints in the statement
of a theorem. For example, we can have,

\begin{lstlisting}[style=lisp]
  (defund foo (x y) (or (implies x y) (and x (not y))))
  (thm
    (implies (hint-wrapper '(:in-theory (enable foo)))
             (foo x y)))
\end{lstlisting}

The \texttt{hint-wrapper} function always returns \texttt{'t}, therefore the
theorem statement is equivalent to \texttt{(foo x y)}.
That it, the hint-wrapper term doesn't change the logical meaning of the theorem 
statement.
However, it provides hints to be used for a computed-hint. A computed-hint is a
program that monitors every subgoal produced during a proof. It recognizes
patterns of a hint-wrapper and installs hints provided by the hint-wrapper into
subsequent goals. In this example, a computed-hint will install the hint
\texttt{:in-theory (enable foo)} into the next subgoal, allowing the theorem to
be proved.

Note that a computed-hint is also a form of proof automation. It allows hints to
be applied intelligently without manual user intervention. However, it is
different from reflection, in that it merely guides the proof and does not
change the theorem statement. Therefore, there isn't any soundness concern with
uses of computed-hints.

\subsection{Enabling a Reflection-base Architecture}
As is discussed in this chapter, clause-processors treat ACL2 proof goals as
quoted terms and permits meta-programming.
Pseudo-evaluators provide a way of verifying soundness of a clause-processor.
Meta-extract allows reflection to be achieved in clause-processors.
Computed-hints allow us to chain several reflection-based steps together to
build an extensible architecture.
Together, the clause-processor, pseudo-evaluator, meta-extract and
computed-hints features enable a reflection-based architecture for Smtlink.

The next section is a detailed description of the architecture of Smtlink. It
shows how the reflection-based architecture is enabled through applications of
clause-processors, pseudo-evaluators, meta-extract and computed-hints.

\section{The Reflection-based Architecture}
\label{sec:arch}
Figure~\ref{fig:arch} shows the Smtlink reflection-based architecture composing
of a chain of verified clause-processors leading to a trusted clause-processor.
The verified steps include user hint processing, user hypotheses introduction,
function expansion, type inference, term replacement, trusted clause-processor
hint generation, and type declaration extraction.
The user can choose between two trusted clause-processors that transliterate
from an ACL2 proof goal to an SMT goal: trusted clause-processor and customized
trusted clause-processor.
The customized trusted clause-processor gives the user the freedom to program
their own solver class using the Z3py interface.
Note that the architecture of Smtlink is chosen to be a linear chain of steps
for simplicity. Smtlink is highly customizable and one can imagine changing the
architecture to introduce branches, and even loops. However, these extensions
are beyond the topic of this thesis.

Each clause-processor serves a certain purpose in the transformation from a
top-level ACL2 proof goal to an SMT term.
A common theme of the chain of verified clause-processors is that they make
heavy use of rule-based reflection. The user can provide rules (established
facts in the ACL2 world) through the Smtlink hint to guide the reflection
transformations achieved in each verified clause processor. The first verified
clause-processor does user hint processing to process the smtlink hints.

For any nontrivial theorem, it is almost always a requirement to provide lemmas
that can help prove the target proof goal. Typically, the user might have proven
a bunch of useful theorems that can help the SMT solver to discharge the proof
goal. Sometimes, providing useful theorems improves the performance of SMT
solving, in other times, it is necessary to know such fact in order to prove the
original goal. Simply providing all lemmas to the SMT solver is a risk for
exponential blowup. The user hypotheses introduction clause-processor allows the
user to introduce only ``interesting'' lemmas to help proving a proof goal.

Often times, users of theorem provers will define functions for proper
abstraction of a nontrivial system. Some of these functions might be recursive.
It would be rather tedious to make the user manually expands proof goals into
SMT primitives. The function expansion clause-processor automatically expands
out functions in a way that is guided by user hints.

To bridge the untyped logic of ACL2 with the many-sorted logic of SMT, Smtlink
does type inference. The type inference clause-processor has two purposes.
First, it assigns a type to each subterm of the proof goal. These types of
subterms ensure that a consistent set of types can be assigned to the ACL2 term.
Second, the type information is passed onto later stages like term replacement
and the final trusted clause-processor for further transformation of the proof
goal.

Further from the problem of the different type system of ACL2 and SMT, functions
in ACL2 are polymorphic whereas functions in Z3 are typed. For example, in ACL2,
\texttt{cons} can be applied to any list, however, in Z3,
\texttt{IntegerList.cons} is the function to apply to an \texttt{IntegerList}.
Smtlink will need to identify types of subterms and replace them with the
appropriately typed versions. For example, we define in ACL2 a function
\texttt{integer-list->cons} and replace \texttt{(cons x y)} with
\texttt{(integer-list->cons x y)}. The term replacement clause-processor also
handles the case when certain sort of SMT doesn't exist in ACL2, which will be
discussed in later sections.

The trusted clause-processor hint generation clause-processor generates hints
that are used by the trusted clause-processor. The last verified
clause-processor does type declaration extraction to extract type predicates
from the disjunct of clauses. At this point, the proof goal is annotated with
necessary hypotheses, fully expanded out, well-typed, and replaced with SMT
corresponding functions. Smtlink then runs the trusted clause-processor to
translate the flattened goal into SMT and invoke the SMT solver.

The rest of this Chapter walks through each of the clause-processors defined in
ACL2. We will show examples that illustrates the purpose of that
clause-processor, what the clause-processor achieves, the main algorithm, and
the technical challenges.

\section{The Clause-processors}
\label{sec:cp}

This section focus on the functionality and technical challenges for each
clause-processor. 

\subsection{User Hint Processing}
Meta-extract allows Smtlink to extract useful rules from the ACL2 logical world
to achieve reflection over the proof goal. However, sometimes Smtlink will need
to reply on the user for specifying what rules are ``interesting''. This is done
by the user providing hints to Smtlink. To distinguish between the ``:hint''
provided to ``defthm'' and the hints provided to ``Smtlink'', we call our hints,
the ``smtlink hint''.

The very first clause-processor processes the hints provided by the user. It
parses the hint syntax into a smtlink hint data structure for internal use.
Table~\ref{tab:smthint} shows hints provided to the four main clause-processors.

\begin{table}[h]
  \begin{tabular}{|r|l|}
    \hline
    User Hypotheses Introduction & Rules for the hypotheses \\
    \hline
    Function Expansion & Returns theorems and expansion level \\
    \hline
    Type Inference & Type inference rules \\
    \hline
    Term Replacement 	& Term replacement rules \\
    \hline
  \end{tabular}
  \caption{Smtlink hints}
  \label{tab:smthint}
\end{table}

A state variable called \texttt{smt-hint} stores a default hint for Smtlink.
When processing the user hint, Smtlink updates the global state. This saves the
user the effort to write repeated hints.
The user hint processing clause-processor doesn't change the proof goal. It only
updates the data structure storing the hints. Therefore the soundness of this
user hint processing clause-processor is trivial.

\subsection{User Hypotheses Introduction}
Smtlink enables the user to add user hypotheses to the target proof goal through
the user hypotheses introduction clause-processor.
For example, in [this proof goal], the user has already proven [this
lemma]. The user might need to inform the SMT solver of this established result.
In order to do that, the user simplify provide this rule in the Smtlink hint.

\begin{lstlisting}[style=lisp]
(the-hint ...)
\end{lstlisting}

Given a proof goal $G$, a user hypothesis rule called $R$ and a substitution for
the hypothesis $Subst$. The user hypotheses introduction clause-processor first
uses meta-extract to extract the body $H$ for rule $R$ from the ACL2 logical
world. It then uses the substitution pair $Subst$ to create a substitution of
the rule $H[Subst]$. Proof goals in ACL2 are represented as a disjunction of
clauses. To add the hypothesis, we add the negation of $H[Subst]$ into the
disjunction of clauses. It can be proved in ACL2 that this transformation is
sound. Adding a list of user hypotheses is an obvious recursion of adding one
user hypothesis.

Note that the user might also want to provide a hypothesis that has not been
proven in ACL2 yet, but they believe it to be true. In that case, Smtlink will
have to return a subgoal for this hypothesis back to ACL2. For simplicity and
performance considerations, Smtlink do not do this currently. However, there is
no theoretical hurdle to add this feature.

\subsection{Function Expansion}

The function expansion clause-processor flattens the proof goal through
beta-reduction. Non-recursive function are expanded out by default. Recursive
functions can be expanded to a user-provided level. The last recursive call is
treated as an uninterpreted function. It is also possible to treat non-recursive
function as uninterpreted function by supplying an expansion level of 0.
Built-in functions and functions for user-defined types are the base case.

\begin{table}[h]
  \begin{tabular}{|r|l|}
    \hline\
    \texttt{equal} & \texttt{\_SMT\_.equal} \\
    \hline
    \texttt{<} & \texttt{\_SMT\_.lt} \\
    \hline
    \texttt{if} & \texttt{\_SMT\_.ifx} \\
    \hline
    \texttt{not} 	& \texttt{\_SMT\_.notx} \\
    \hline
    \texttt{binary-+} 	& \texttt{\_SMT\_.plus} \\
    \hline
    \texttt{binary-*} 	& \texttt{\_SMT\_.times} \\
    \hline
    \texttt{unary-/} 	& \texttt{\_SMT\_.reciprocal} \\
    \hline
    \texttt{unary--} 	& \texttt{\_SMT\_.negate} \\
    \hline
  \end{tabular}
  \caption{Built-in Functions}
  \label{tab:builtin}
\end{table}

Table~\ref{tab:builtin} shows the built-in functions that operates over built-in
types \texttt{booleanp}, \texttt{integerp}, \texttt{rationalp} and
\texttt{symbolp}. For user-defined algebraic datatypes, constructor and
destructor functions are also treated as base functions. For alist types,
\texttt{assoc-equal} and \texttt{acons} are base functions.

Given the input proof goal $G_{hypo}$, the function expansion clause-processor
recursively traverses the goal. If it encounters a built-in function or a
function associated with a user-defined type, it leaves the term as is.
Otherwise, if it is a non-recursive function, the clause-processor uses
meta-extract to extract the equivalence rule for the function definition from
the ACL2 logical world. It then tries to unify the input argument with the
formals of the function definition to get a substitution. Using the
substitution, the clause-processor does beta-reduction to fully expand out the
function. If the function is non-recursive, the function will be expanded out to
a given level. Lambdas are fully expanded out as well.

One technical challenge for the function expansion clause-processor is its
termination proof. We use a clock that always decrease to force the termination
of the expansion. The clock should be large enough for the clause-processor to
be useful. We choose the total number of function definitions in the logical
world as the clock.

\subsection{Type Inference}
Suppose the proof goal after function expansion is called $G_{expand}$. This
goal is composed of constants of built-in types, built-in function calls and
user-defined uninterpreted functions. This goal is ready for translation to SMT
if all constants and functions have a correspondence in the SMT solver.
However, there are many differences between the logic of ACL2 and SMT that makes
it impossible do a direct translation.

First, the logic of ACL2 is untyped. All functions in ACL2 are total functions.
For example, \texttt{car} takes any type of input argument, an integer list, a
rational list, or even a symbol \texttt{'a}. However, there is no such function
in the SMT solver. The SMT solver uses many-sorted logic. For an integer list,
its \texttt{car} function is typed as \texttt{IntegerList.car}. This means that
for any term \texttt{x}, we need to know what its type is, to be able to
recognize the sort for the function that operates on it. For example, if we can
infer that \texttt{x} is of \texttt{integer-list-p}, then we know \texttt{(car
  x)} could be translated into \texttt{IntegerList.car}.

In addition, there exists subtle differences between the logic of ACL2 and the
logic of SMT. For example, \texttt{(car x)} is \texttt{'nil} when x is
\texttt{'nil}. However, \texttt{IntegerList.car(IntegerList.nil)} returns an
arbitrary integer. This means that, we can only translates \texttt{(car x)} into
\texttt{(IntegerList.car(x))} when \texttt{(not (equal (x nil)))}.

Smtlink does type inference to solve these issues. To handle these problems, the
type inference of Smtlink should work for not just regular types like integers,
rationals, booleans, but also for the other SMT sorts like datatypes, abstract
sorts and array sorts. In addition, it needs to do a bit more reasoning about
checking for \texttt{'nil} or \texttt{0}.

The logic of ACL2 is untyped, so the question is where do the types come from.
Here we identify a set of functions in ACL2 as type predicates:
built-in types like \texttt{integerp}, \texttt{rationalp}(\texttt{realp} for
ACL2(r)), \texttt{booleanp}, \texttt{symbolp}, and user-defined types like
alists, algebraic datatypes and abstract types. (We identify for each of these
types an SMT sort.)
For each of these types, several functions that operates over these types are
identified. Each such function is associated with at least one type inference
rule. The Smtlink hint for type inference allows the user to provide hints about
these type inference rules. To allow reasoning about simple constraints over
types, we also identify any single-argument function as part of the type
information. For example, for a variable \texttt{x}, its type information can be
\texttt{(and (integerp x) (x < 0))}, which essentially states that x is a
negative integer.

Given that a set of functions recognized as type predicates, and a set of type
inference rules, Smtlink uses a rule-based two-pass algorithm for type
inference.

We define a data structure called \texttt{typed-term}.

\begin{lstlisting}[style=lisp]
  (defprod typed-term ((term pseudo-termp)
                       (path-cond pseudo-termp)
                       (judgements pseudo-termp)
                       (smt-judgements pseudo-termp)))
\end{lstlisting}

The \texttt{term} field represents a clause or a subterm of the clause.
\texttt{path-cond} is the conjunction of branch conditions that leads to
\texttt{term}, \texttt{judgements} are all the type judgements inferred for
\texttt{term} and \texttt{smt-judgements} is one consistent type judgements
chosen for each subterm. For example, if a goal includes the hypothesis
\texttt{integerp x}, and has the subterm \texttt{(* x x)}, the typed-term for
\texttt{(* x x)} could be,

\begin{lstlisting}[style=lisp]
  (typed-term '(binary-* x x) '(integerp x)
    '(if (if (integerp (binary-* x x)) 't 'nil)
         (if (if (integerp x) 't 'nil)
             (if (if (integerp x) 't 'nil) 't 'nil)
             'nil)
         'nil)
    '(if (if (integerp (binary-* x x)) 't 'nil)
         (if (if (integerp x) 't 'nil)
             (if (if (integerp x) 't 'nil) 't 'nil)
             'nil)
         'nil))
\end{lstlisting}

The shape of the type judgements maps directly to the shape of the
term\footnote{Note that for each occurrence of \texttt{x} in \texttt{(binary-* x
    x)}, there is a type judgement for it}. The judgement is added as a
hypothesis to the main goal. We provide a set of constructor and destructors
that operate over \texttt{typed-term} that make is easy to access subterms,
their path conditions and type judgements.

The two-pass type inference algorithm is implemented using two clause
processors. The first clause processor performs a post-order traversal of the
goal, and constructs a \texttt{typed-term} that annotates each subterm with
\texttt{all} type recognizers (given the provided list of type inference rules)
that it can be shown to be satisfied by the subterm. A set of supertype and
subtype rules can be provided to expand the set of type recognizers inferred.
This clause processor recursively traverses the pseudo-term with the following
four cases:

\begin{enumerate}
  \item Constants: Types of constants are trivially proved using
    \texttt{meta-extract}.
  \item Variables: We find type judgements from the path condition. The
    disjuncts of a clause are sorted so that the type predicates appear early in
    the path.
  \item \texttt{if}-expression: We treat \texttt{if} specially because the
    condition of an \texttt{if} is used to augment the path condition for the
    ``then'' and ``else'' arguments.
  \item Function calls (other than \texttt{if}): Type inference rules are
    obtained using \texttt{meta-extract}. Polymorphism is supported by allowing
    the same function to have multiple type inference rules.
\end{enumerate}

Note that this post-order type inference algorithm is not just limited to the
use within Smtlink. It can be used for any type inference within ACL2. Even
better the user can control the way they want type inference to be carried out
through enabling the right set of rules.

Having found all valid type judgements for each subterm, the second clause
processor chooses type judgements that are consistent for the many-sorted logic
of the SMT solver. For example, \texttt{(list 1 2 3)} satisfies both
\texttt{integer-listp} and \texttt{rational-listp}, but Z3 considers these to be
distinct sorts. Note that this clause processor chooses by discarding type
judgements that won't be used in the translation. Therefore it must be sound.

\subsection{Term Replacement}

Term replacement replaces subterms in the proof goal to another term that is
equivalent to this term given the current path-condition. There are two use
cases in Smtlink that requires term replacement. However, the term replacement
clause processor is a general purpose rule-based term transformer. It can be
used for any kind of term replacement, given that the term replacement rule
follows a given structure.

The first usage is for static dispatching of polymorphic functions. For example,
the lists in ACL2 shares the same set of constructor and distructors
\texttt{cons}, \texttt{car} and \texttt{cdr}. However, we want to be able to
translate them into the their typed versions in Z3, for example,
\texttt{IntegerList.cons}, \texttt{IntegerList.car} and
\texttt{IntegerList.cdr}. This translation should also solve the problem that
\texttt{IntegerList.car} means different things from \texttt{car} when the input
is an empty list. Imagine given the following typed-term:

\begin{lstlisting}[style=lisp]
  (typed-term
    '(car x lst)
    '(if (integerp x)
         (if (integer-list-p lst) (not (null lst)) 'nil)
     'nil)
    '(if (if (integerp (car x lst)) 't 'nil)
         ...
         'nil)
    '(if (if (integerp (car x lst)) 't 'nil)
         ...
         'nil))
\end{lstlisting}

Given that we have defined a typed version of \texttt{car} called
\texttt{integer-list-car} that has the following definition:

\begin{lstlisting}[style=lisp]
  (defun integer-list-car (x)
    (if (null x) 0 (car x)))
\end{lstlisting}

we can prove the following theorem:

\begin{lstlisting}[style=lisp]
  
\end{lstlisting}


\subsubsection{Alists vs Arrays}

\subsection{Trusted Clause-processor Hint Generation}

\subsection{Type Declaration Extraction}

\subsection{The Trusted Clause-processor}
We start by identifying a subset of the SMT sorts that are supported by Smtlink.
These include Bool sort, Int sort, Real sort, Datatype sort, Array sort, and
Abstract sort. A set of functions are associated with each sort. These sorts and
their associated functions are embedded in ACL2. For each sort and its
associated functions, properties that these SMT sorts should satisfy are
established within ACL2. These properties ensure that the last step of direct
translation from ACL2 to Z3 is a sound translation. The rest of the architecture
takes care of transforming from an ACL2 term to an embedding of the SMT term.
These transformations allow the user to define an ACL2 proof goal on the
high-level and the tool will do the heavy work of transforming it into an SMT
term.

\subsection{Connecting the Clause-processors}

The chain of verified clause-processors are connected through use of
hint-wrapper and computed-hints.
We define a ``hint-wrapper'' function that always returns \texttt{nil} called
\\
\texttt{SMT::hint-please}.
The hint-wrapper term is added as a disjunct to a proof goal.
\texttt{SMT::hint-please} only takes one input argument -- the list of hints.
A computed-hint called
\texttt{SMT::SMT-computed-hint} searches for a disjunct of the form
\texttt{(SMT::hint-please ...)} in each goal generated by Smtlink.
When it finds such an instance, the \texttt{SMT::SMT-computed-hint}
will return a \texttt{computed-hint-replacement} of the form:
\begin{lstlisting}[style=Lisp]
  `(:computed-hint-replacement
    ((SMT::SMT-computed-hint clause))
    (:clause-processor (SMT::some-verified-cp clause ,combined-hint)))
\end{lstlisting}
This applies the next verified clause-processor called
\texttt{SMT::some-verified-cp} to the current subgoal and installs the
computed-hint \texttt{SMT::SMT-computed-hint} on subsequent subgoals again.
The \texttt{SMT::some-verified-cp} clause-processor is one step in a sequence of
verified clause processors.
\smtlink{} uses a configuration table called \texttt{*smt-architecture*} to
specify the sequence of clause processors, as shown in Figure~\ref{fig:arch}.
Each clause processor consults this table to determine its successor.

\begin{figure}\begin{center}
  \resizebox{1.0\textwidth}{!}{\chapter{The Reflection-based Integration Architecture}
\label{ch:arch}

% There doesn't exist a direct correspondence of the alist type in Z3. One of the
% idea is to use \texttt{ArraySort} to represent alists. However, there are
% several difficulties.
% First, ACL2 alists can be empty but Z3 arrays are infinite. We solve this
% problem by using constant arrays initialized with \texttt{nil} constants to
% represent empty alists.
% Second, the destructor function \texttt{assoc-equal} doesn't correspond directly
% to Z3's \texttt{Select} function. More specifically, \texttt{assoc-equal}
% produces a cons-pair, while \texttt{Select} produces a value.
% To solve these problems, for each alist type, we provide a corresponding array
% definition in ACL2.
% As is described in Chapter~\ref{ch:arch}, the replacement clause-processor
% replaces uses of alist recognizers, empty alists, \texttt{acons},
% \texttt{assoc-equal} and \texttt{cdr} with their corresponding array functions.

}
  \caption{The Smtlink Architecture}\label{fig:arch}
\end{center}\end{figure}

For example, after Smtlink finishes processing user hints, it then consults the
configuration table for the next clause-processor, which is ``CP1'' for adding
user hypotheses.
A \texttt{SMT::hint-please} term is added to the list of clauses informing
the computed-hint what clause-processor to use for the next step.
The computed-hint will then be able to install the next step for adding user
hypotheses. Likewise, at the end of ``CP1'', a \texttt{SMT::hint-please} term is
added to the list of clauses, and it again informs the computed-hint what
clause-processor hint to generate.
Note that this architecture is configured by an alist table that can be easily
extended and reconfigured.

\section{Summary}
\label{sec:summary}

\end{comment}