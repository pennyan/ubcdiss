\chapter{Soundness}
\label{ch:Soundness}

\section{The Soundness Argument}\label{sec:SoundArg}
Given the original goal $G$, \smtlink{} uses a series of verified
clause-processors to transforms $G$ into a goal $\Gtcp$ that can be readily
translated into \zthree{}.
Clause processors are programs that transform an input clause into a conjunction
of output clauses.
When a clause processor is verified, the theorem that the conjunction of output
clauses imply the input clause is established.
Therefore, we know $\Gtcp$ implies the original goal $G$ and the transformation
steps are sound.
After the verified clause-processors, a final trusted clause-processor
translates $\Gtcp$ into \zthree{}. We refer to this translated form as $\Gsmt$.

Let $x_1$, $x_2$, \ldots, $x_n$ denote the free variables in $\Gtcp$,
let $\Gsmt$ denote the translated goal,
and $\tilde{x}_1$, $\tilde{x}_2$, \ldots, $\tilde{x}_n$ denote
the free variables of $\Gsmt$.

For soundness, we want
\begin{equation}\label{eq:soundness}\begin{array}{rcl}
\mathrm{SMT} \vdash \Gsmt &\Rightarrow& \mathrm{ACL2} \vdash \Gtcp
\end{array}\end{equation}
In the remainder, we assume
\begin{itemize}
\item ACL2 and the SMT solver are both sound for their respective theories.
\item The SMT solver is a decision procedure for a decidable fragment of first-order logic.
  In particular, this holds for Z3, the only SMT solver that is currently
  supported by \smtlink{}. In addition, we are working with a quantifier-free
  fragment of Z3's logic.
\item There is a one-to-one correspondence between the free variables of $\Gtcp$
  and the free variables of $\Gsmt$.
  This is the case with the current implementation of \smtlink{}.
\end{itemize}

\begin{theorem}
  for each model in ACL2, there exists a model in Z3
\end{theorem}
Now, suppose that $\Gtcp$ is not a theorem in ACL2.  Then, by G\"{o}del's
Completeness Theorem, there exists a model of the ACL2 axioms that satisfies
$\neg \Gtcp$. We need to show that in this case there exists a model of the SMT
solver's axioms that satisfies $\neg \Gsmt$.  There are two issues that we must
address. First, we need to provide, for the interpretation of any function
symbol $\facl$ in $\Gtcp$, an interpretation for the corresponding function
symbol $\fsmt$ in $\Gsmt$.
This brings us to the second issue: the logic of ACL2 is untyped, but the logic
of SMT solvers including Z3 is many-sorted.
Thus, there are models of the ACL2 axioms that have no correspondence with the
models of the SMT solver.
We restrict our attention to goals, $\Gtcp$ where the type of each subterm in the
formula can be deduced.  We refer to such terms as translatable.  If $\Gtcp$ is
not translatable, then \smtlink{} will fail to prove it.

\begin{definition}
  translatable's definition
\end{definition}

For the remainder, we restrict our attention to translatable goals.
Because $\Gtcp$ is translatable, there is a set $R$ of unary recognizer functions
(primitives such as rationalp that return a boolean) and also a set $S$ of other
functions, such that every function symbol in $\Gtcp$ is a member of $R$ or of $S$,
and every function in $S$ is ``well-typed'' with respect to $R$ in some sense that
we can define roughly as follows.
We associate each function symbol $\facl$ in $S$ with a function symbol $\fsmt$
of Z3, and each predicate $r$ in $R$ with a type in Z3.
The trusted clause processor checks that there is a ``type-hypothesis'' associated
with every free variable of $\Gtcp$ and a fixing function for every type-ambiguous
constant (e.g.\ \texttt{nil}) -- $\Gtcp$ holds trivially if any of these
type-hypotheses are violated.
For every function $\facl$ in $S$, we associate a member of $R$ to each of its
arguments (i.e.\ a ``type'') and also to the result.
% This also applies to function
% \texttt{if} where its two branches have to satisfy the same function from $R$.
For user-defined functions (i.e.\ uninterpreted function for the SMT solver),
\smtlink{} generates a subgoal for each call to $\fsmt$: if the arguments satisfy
their declared types (i.e., predicates from $R$), then the result must satisfy its
declared type as well.
For built-in ACL2 functions (e.g.\ \texttt{binary-plus}) we assume the ``obvious''
theorems are present in the ACL2 logical world.
Now suppose we have a model, $M_1$, of $\neg \Gtcp$, and consider the submodel,
$M_2$, containing just those objects $m$ such that $m$ satisfies at least one
predicate in $R$ that occurs in $\Gtcp$. Note that $M_2$ is closed
under (the interpretation of) every operation in $S$, because $\neg \Gtcp$
implies that all of the ``type-hypotheses'' of $\Gtcp$ are true in $M_1$. This
essentially excludes ``bad atoms'', as defined by the function
\texttt{acl2::bad-atom}.
Then because $\Gtcp$ is quantifier-free, $M_2$ also satisfies $\neg \Gtcp$.
We can turn $M_2$ into a model $M_2'$ for the language of Z3, by assigning the
appropriate type to every object.  (As noted in Section~\ref{sec:reals}, $M_2'$
satisfies the theory of Z3 if $M_2$ is a model of ACL2(r); but for ACL2 that is
not the case, so in future work, we expect to construct an extension of $M_2'$
that satisfies all of the axioms for real closed fields.)
Then we have the claim: for every assignment $s$ from the free
variables of $\Gtcp$ to $M_2$ with corresponding typed assignment $s'$ from the
free variables of $\Gsmt$ to $M_2'$ , if $\neg \Gtcp$ is true in $M_2$
under $s$, then $\neg \Gsmt$ is true in $M_2'$ under $s'$.
Thus, if $\Gtcp$ is translatable, and $\neg \Gsmt$ is unsatisfiable, we conclude
that $\Gtcp$ is a theorem in ACL2.

In the rest of this section, we discuss for each of the recognizer functions and
each of the basic functions in ACL2, how we associate them with the
corresponding Z3 functions

\section{Soundness for Uninterpreted Functions}\label{sec:SoundUninterpreted}

\section{Soundness for Types}\label{sec:SoundType}

\section{The Trusted Clause Processor}\label{sec:SoundTrusted}

\endinput

\subsection{Booleans, Integers, Rationals, and Reals}\label{sec:reals}
If a term is a boolean constant, then the translation to the SMT solver is direct.
Likewise, if $x_i$ is free in $\Gtcp$ and \texttt{(booleanp $x_i$)}
is one of the hypotheses of $\Gtcp$, then $\Gtcp$ holds trivially in the case that
$x_i \not\in \{\texttt{t},\,\texttt{nil}\}$.  Thus, in $\Gsmt$ \smtlink{} can
represent the hypothesis \texttt{(booleanp $x_i$)} with the declaration\\
\rule{2em}{0ex}\texttt{x\_i = Bool('x\_i')}\\
without excluding any satisfying assignments.  We assume that the boolean operations
of the SMT solver (e.g. \texttt{And}, \texttt{Or}, \texttt{Not}) correspond
exactly to their ACL2 equivalents when their arguments are boolean.  If a boolean
operator is applied to a non-boolean value, then Z3 throws an exception, and we
regard $G$ as non-translatable.

Similar arguments apply in the case that $x_i$ is an integer, rational, or real
number. We represent ACL2 rational numbers as Z3 real numbers. Because every
rational number is a real number, any satisfying assignment of rational numbers
to rational variables in $\neg \Gtcp$ has a corresponding assignment for $\neg
\Gsmt$. Thus, $\Gsmt$ is a generalization of $\Gtcp$.
We note that for ACL2, formally proving the soundness of this generalization
requires extending our previously discussed $M_2'$ model into a model that
satisfies the theory of real closed
fields~\footnote{http://smtlib.cs.uiowa.edu/theories-Reals.shtml}, because we
are translating rationals in ACL2 to reals in Z3. We haven't wrapped our heads
around how to do that extension in a many-sorted setting, therefore we designate
this to be future work. As with booleans, we assume that arithmetic and
comparison operators have equivalent semantics in ACL2 and the 
SMT solver. In fact, we use the Python interface code to enforce this
assumption. As an example, ACL2 allows the boolean values \texttt{t} and
\texttt{nil} to be used in arithmetic expressions -- both are treated as 0.
Z3 also allows \texttt{True} and \texttt{False} to be used in integer
arithmetic, with \texttt{True} treated as 1 and \texttt{False} treated as 0.
To ensure that $\Gsmt \Rightarrow \Gtcp$, our Python code checks the sorts of the
arguments to arithmetic operators to ensure that they are integers or reals, where
the interpretations are the same for both ACL2 and Z3.

When \smtlink{} is used with ACL2(r), non-classical functions are non-translatable.
We believe that if $\neg \Gtcp$ is classical and satisfiable,
then there exists a satisfying assignment to $\neg \Gtcp$ where all real-valued
variables are bound to standard values. We believe the sketched proof in the
beginning of Section~\ref{sec:theories} works well for ACL2(r). If we are wrong,
we hope that one of the experts in non-standard analysis at the workshop will
correct us.

\subsection{Symbols}
A very important basic type in ACL2 is \texttt{symbolp}. We represent symbols
using an algebraic datatype in Z3. In the z3 interface class, we define a
\texttt{Datatype} called \texttt{z3sym}, with a single field of type
\texttt{IntSort}. Symbol variables are defined using the datatype
\texttt{z3sym}. We then define a class called \texttt{Symbol}. This class
provides a variable \texttt{count} and a variable \texttt{dict}. It also
provides a function called \texttt{intern} for generating a symbol constant.
This class keeps a dictionary mapping from symbol names to the generated
\texttt{z3sym} symbol constants. This creates an injective mapping from symbols
to natural numbers. 
All symbol constants that appeared in the term are mapped onto the first
several, distinct, naturals.

If a satisfying assignment to $\neg \Gtcp$ binds a symbol-valued variable
to a symbol-constant that doesn't appear in $\Gtcp$, then in our soundness
argument, we construct a new symbol value for $\neg \Gsmt$ using an integer
value distinct from the ones used so far -- we won't run out.
Thus, all symbol values in a satisfying assignment to $\neg \Gtcp$
can be translated to corresponding values for $\neg \Gsmt$.
The only operations that we support for symbols are comparisons for equality or
not-equals.  We assume that these operations have corresponding semantics in
ACL2 and the SMT solver.


\subsection{FTY types}
We have added support for common \texttt{fty} types that enable \smtlink{} to
automatically construct bridges from the untyped logic of ACL2 to the typed logic of Z3.
Currently, \smtlink{} infers constructor/destructor relations and other properties
of these types from the \texttt{fty::flextypes-table}.
Thus, the use of \texttt{fty} types extends
the trusted code to include the correctness of these tables.  This trust is mitigated
by two considerations.  First, \smtlink{} only uses \texttt{fty} types that have been
specified by the user in a hint to the \smtlink{} clause processor.  If the user
provides no such hints, no \texttt{fty} types are used by \smtlink{}, and no
soundness concerns arise.

Second, we expect that the information that \smtlink{} obtains from these tables could
be obtained instead from the ACL2 logical world using \texttt{meta-extract} in
\smtlink{}'s verified clause-processor chain.  We see the current implementation
as a useful prototype to explore how to seamlessly infer type information from
code written according to a well-defined type discipline.

\subsubsection{fty::defprod}
The algebraic datatypes of Z3 correspond directly to \texttt{fty::defprod}.
\smtlink{} simply declares a Z3 datatype with a single constructor whose
destructor operators are the field accessors of the product type.
\smtlink{} requires that the arguments to the \texttt{fty} constructors satisfy
the constructors' guards -- otherwise $\Gtcp$ is non-translatable.
The only operations on product types are field accessors, i.e.\ destructors.
For translatable terms, the SMT type has the same construct/destructor theorems
as the FTY type.  Thus, the \smtlink{} translation maintains equivalence
constructors and field accessors of product types.

\subsubsection{fty::deflist}
Lists are essentially a special case of a product type.  For example,\\
\noindent\begin{minipage}[t]{.35\textwidth}\label{prog:deflist}
\begin{lstlisting}[caption=ACL2 deflist,frame=tlrb,style=snippet,language=LISP]{ACL2 deflist}
(deflist integer-list
  :elt-type integerp
  :true-listp t)
\end{lstlisting}
\end{minipage}\hfill
\begin{minipage}[t]{.61\textwidth}
\begin{lstlisting}[caption=Z3 Datatype,frame=tlrb,style=snippet,language=Python]{Z3 Datatype}
integer_list= z3.Datatype('integer_list')
integer_list.declare('cons',
                     ('car', _SMT_.IntSort()),
                     ('cdr', integer_list))
integer_list.declare('nil')
integer_list = integer_list.create()
def integer_list_consp(l):
  return Not(l == integer\_list.nil)
\end{lstlisting}
\end{minipage}
ACL2 overloads \texttt{cons}, \texttt{car}, and \texttt{cdr} to apply to any list.
In contrast, Z3's typed logic has a separate \texttt{cons}, \texttt{car}, \texttt{cdr}
functions for each list type.
This is why our examples from Section~\ref{sec:expl} require fixing functions to convey
the type information to the trusted-clause processor.
We believe that most of the users' burden of typed lists will be removed in a future
release by adding type-inference to \smtlink{}. There are soundness issues that
must be addressed. In ACL2, \texttt{(equal (car nil) nil)}.
In Z3,\\
\rule{2em}{0ex}\texttt{integer\_list.car(integer\_list.nil)}\\
is an arbitrary integer.
To ensure soundness, the trusted-clause processor produces
the proof obligation \texttt{(consp x)} for every occurrence of \texttt{(car x)}
that it encounters.
Under this precondition, the Z3 translation preserves the constructor/destructor
relationship for lists.
Likewise\\
\rule{2em}{0ex}\texttt{integer\_list.cdr(integer\_list.nil)}\\
is an arbitrary \texttt{integer\_list}.
Thus, \smtlink{} enforces the \texttt{:true-listp t} declaration for list types.
Because ``arbitrary'' includes \texttt{integer\_list.nil} in addition to all other \texttt{integer\_list}s,
these construction
ensures that the SMT solver can choose the value for \texttt{integer\_list.cdr($\tilde{\texttt{x}}$)} for $\Gsmt$
that corresponds to the value of \texttt{(cdr x)} for any assignments for $\Gtcp$.
The $\Gsmt$ is a generalization of $\Gtcp$.

\subsubsection{fty::defalist}
\smtlink{} represents alists with SMT arrays.  We only support the operations
\texttt{acons} and \texttt{assoc-equal} for alists.  Then we have:
\begin{lstlisting}[style=snippet]
(defthm alist-axioms
  (implies (not (equal key1 key2))
           (and (equal (assoc key1 (acons key1 value alist)) value))
                (equal (assoc key1 (acons key2 value alist)) (assoc key1 alist)))
                (equal (assoc key1 nil) nil))
\end{lstlisting}
The corresponding theorem in the theory of arrays is
\begin{lstlisting}[style=snippet]
(defthm array-axioms
  (implies (not (equal addr1 addr2))
           (and (equal (load addr1 (store addr1 value array)) value))
                (equal (load addr1 (store addr2 value array)) (load addr1 array))))
\end{lstlisting}
Note that \smtlink{} does not support operations such as \texttt{cdr},
\texttt{nth}, \texttt{member}, or \texttt{delete-assoc} that would
``remove'' elements from an alist. Also, Z3 arrays are typed.

The key issue in the translation is how to handle the case when a key is not found in the
alist (ACL2) or array (SMT).
Our solution is to make the element type of the SMT array be an option type called
\texttt{keyType\_elementType}.  This type can either be a \texttt{(key, value)} tuple
or \texttt{keyType\_elementType.nil}.
Thus, any value returned by \texttt{assoc-equal} with proper alist and key types
has a corresponding \texttt{keyType\_elementType} value.  Thus, any value for an
\texttt{assoc-equal} terms in $\Gtcp$ can be represented in $\Gsmt$.

When applying \texttt{cdr} to a \texttt{keyType\_elementType}, we must ensure that the
\texttt{keyType\_elementType} value is not nil.  This is analogous to the issue with lists:
in ACL2, \texttt{(equal (cdr nil) nil)} but in Z3,\\
\rule{2em}{0ex}\texttt{keyType\_elementType.cdr(keyType\_elementType.nil)}\\
is an arbitrary value of \texttt{elementType}.
Thus, the trusted-clause processor produces the proof obligation for ACL2
\texttt{(not (null x))} for every occurrence of \texttt{(cdr x)} when \texttt{x} is the
return value from \texttt{assoc-equal}.
Under this precondition, \texttt{cdr} is only applied to non-nil values from
\texttt{assoc-equal} and we maintain correspondence of values for terms in
$\Gtcp$ and $\Gsmt$. 

By only providing \texttt{acons} and \texttt{assoc-equal}, the \smtlink{}
support for alists is rather limited.  Nevertheless, we have found it to be very
useful when reasoning about problems where alists are used as simple
dictionaries. 

\subsubsection{fty::defoption}
As is shown in Program~\ref{prog:defoption}, the translation of
\texttt{defoption} is straightforward.

\noindent\begin{minipage}{.35\textwidth}\label{prog:defoption}
\begin{lstlisting}[caption=ACL2 deflist,frame=tlrb,style=snippet,language=LISP]{ACL2 defoption}
(defoption maybe-integer
           integerp)


\end{lstlisting}
\end{minipage}\hfill
\begin{minipage}{.61\textwidth}
\begin{lstlisting}[caption=Z3 Datatype,frame=tlrb,style=snippet,language=Python]{Z3 Datatype}
maybe_integer= z3.Datatype('maybe_integer')
maybe_integer.declare('some', ('val', IntSort()))
maybe_integer.declare('nil')
maybe_integer = maybe_integer.create()
\end{lstlisting}
\end{minipage}

In this example, the \texttt{maybe-integer-p} recognizer maps to
\texttt{maybe_integer} type. The constructor \texttt{maybe-integer-some} maps to
\texttt{maybe_integer.some}. The destructor \texttt{maybe-integer-some->val} 
maps to \texttt{maybe_integer.val}. The none type \texttt{nil} maps to
\texttt{maybe_integer.nil}. Typical users of FTY types won't write
\texttt{maybe-integer-some} constructor and \texttt{maybe-integer-some->val}
destructors. They will first check if a term is nil, and then assume the term is
an \texttt{integerp}. When a program returns an \texttt{integerp}, ACL2 knows it
is also a \texttt{maybe-integerp}. Due to lack of type inference capabilities,
\smtlink{} currently requires the user to use those constructors, therefore
maintaining the option type through function calls where a
\texttt{maybe-integerp} is needed and use the destructors where an
\texttt{integerp} is needed.  The constructor function satisfies the same theorems
in ACL2 and Z3.  Therefore, it's sound.  For the field-accessor, when trying to
access field of a \texttt{nil}, ACL2 returns the fixed default value, while Z3
will return arbitrary value of that \texttt{some} type. The Z3 values include
the ACL2 value.  \texttt{nil} is trivially the same. Thus, the \smtlink{}
translation maintains equivalence of values of terms for constructors and field
accessors of option types.

\subsection{Uninterpreted Functions}
The user can direct \smtlink{} to represent some functions as uninterpreted functions
in the SMT theories.  Let \texttt{(f arg1 arg2 \ldots argk)} be a function instance
in $\Gtcp$.  \smtlink translates this to\\
\rule{2em}{0ex}\texttt{f\_smt(arg\_smt1, arg\_smt2, \ldots, arg\_smtk)}\\
The constraints for \texttt{f\_smt} are the types of the argument, the type of the
result, and any user-specified constraints.
If the function instance in $\Gtcp$ violates the argument type constraints, then
the term is untranslatable -- currently, \smtlink{} produces an SMT term
that provokes a \texttt{z3types.Z3exception}.
For each function instance in $\Gtcp$ that is translated to an uninterpreted function,
\smtlink{} produces a proof obligation for ACL2 that the function instances in
$\Gtcp$ satisfy the given type-recognizers.  Likewise, if the user specified
any other constraints for this function, they are returned as further ACL2 proof
obligations.
Under these preconditions, any value that can be produced by \texttt{f} satisfies the
constraints for \texttt{f\_smt}.
Thus, we maintain correspondence of values for terms in $\Gtcp$ and $\Gsmt$.

% \subsection{Soundness Wrap-Up}
% Terms in SMT solvers such as Z3 are typed.  \smtlink{} support booleans, integers,
% reals and rationals, symbols, product types, lists, alists, and option types.  For
% each of these, we have shown that for all operations supported for each class of
% types, terms in $\Gsmt$ can have the values corresponding to those of $\Gtcp$.
% More generally, $\Gsmt$ can be a generalization of $\Gtcp$ by allowing terms of
% $\Gsmt$ to have values that the corresponding terms of $\Gtcp$ cannot.
% In cases where ACL2 and Z3 interpretations of terms could diverge, the
% trusted clause processor generates preconditions that are verified by ACL2 to ensure
% that the functions in question are only applied to terms for which their
% ACL2 and Z3 behaviors coincide.
% The key to soundness is that if there exists a valuation of the free variables
% of $\Gtcp$ such that $\Gtcp$ is false, then there is a corresponding valuation
% of the free variables of $\Gsmt$ such that $\Gsmt$ is false.  Thus, if there
% is a counter-example for $\Gtcp$, then $\neg\Gsmt$ is satisfiable.  We trust
% the SMT solver to find a satisfying assignment to $\Gtcp$ if one exists (or
% report ``unknown'').  Thus, if the SMT solver reports ``unsat'', we conclude
% that $\neg\Gsmt$ is unsatisfiable; therefore, $\neg\Gtcp$ is unsatisfiable, and
% the original goal is indeed a theorem.
