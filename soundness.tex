\chapter{Soundness}
\label{ch:soundness}

\smtlink{} is composed of several verified reflection-based clause-processors
that transforms the original goal into a form that can be directly translated
into Z3.
Then \smtlink{} uses a trusted clause-processor for transliterating from the
language of ACL2 to the language of Z3 with the Python interface.
This chapter focus on discussing the soundness of Smtlink.

\section{Assumptions}\label{sec:soundness:assumpt}
The soundness of Smtlink is based on the following assumptions:

\begin{assumption}\label{sec:soundness:assumpt1}
  ACL2 and the \acs{SMT} solver are both sound for their respective
  theories.
\end{assumption}
\smtlink{} trust both ACL2 and Z3.

\begin{assumption}\label{sec:soundness:assumpt2}
  If the solvers used by the SMT solver on translated ACL2 formulas
  successfully verify a formula (i.e. they show that (not claim-smt)
  is unsatisfiable), then the claim is a theorem of the axioms for the
  many-sorted first-order logic of SMTLIB
\end{assumption}
In addition, we are working with a quantifier-free fragment of Z3's
logic.

\begin{assumption}
  There is a one-to-one correspondence between the free variables of $\Gtcp$
  and the free variables of $\Gsmt$.
\end{assumption}
This is the case with the current implementation of \smtlink{}. The
transliteration ensures that there is no naming collision.

\begin{assumption}
  There is a one-to-one correspondence between the functions of $\Gtcp$
  and the functions of $\Gsmt$.
\end{assumption}

\begin{assumption}
  The SMT solvers reject any term that is not well-sorted.
\end{assumption}

A verified clause-processor ensures that the conjunction of output clauses
implies the input clause.
Given Assumption~\ref{sec:soundness:assumpt1}, assuming ACL2 is sound, the
pipeline of verified clause processors are sound.
However, an intrepid user might choose to use the trusted clause-processor by
itself. Therefore it is crucial that the soundness argument for the trusted 
clause-processor stands on its own.

In this chapter, we focus on the last trusted clause-processor and discuss how
we formulate the soundness argument for this last step.


\section{The Top-Level Proof}
Now, let $G$ denote the original goal, $\Gtcp$ denote the input goal to the
trusted clause-processor, and $\Gsmt$ denote the transliterated \acs{SMT} goal.
Let $x_1$, $x_2$, \ldots, $x_n$ denote the free variables in $\Gtcp$, and
$\tilde{x}_1$, $\tilde{x}_2$, \ldots, $\tilde{x}_n$ denote the free variables of
$\Gsmt$.
As described in Section~\label{sec:soundness:assumpt}, for each free variable in
$\Gtcp$, there is an associated type-recognizer that we denote as $T(x_i)$.
$\Gtcp$ is a theorem if it holds in all models of the first-order logic of ACL2.
Suppose that $\Gtcp$ is not a theorem in ACL2.  Then, by G\"{o}del's
Completeness Theorem, there exists a model of the ACL2 axioms that satisfies
$\neg \Gtcp$.
If we can show that a model of the theory of ACL2 for $\neg \Gtcp$ implies a
model of the theory of the SMT solver for $\neg \Gsmt$, then, by
Assumption~\ref{sec:soundness:assumpt1} we establish the soundness of the
trusted clause processor. 

We formulate the main claim as the following:

\emph{CLAIM} Assume that $\Gtcp$ is a (quantifier-free) formula of ACL2 that is
   well-typed.  Also assume that its translation $\Gsmt$ is verifiable by
   the intended SMT solvers.  Then $\Gtcp$ is a theorem of ACL2.
\\

\emph{Proof sketch} ``SMT axioms'' refers to the axioms alluded to
in Assumption~\ref{sec:soundness:assumpt2}, and ``well-typed'' means that
necessary ACL2 theorems (see Section~\ref{sec:soundness:sorts}) have
   been proved to permit suitable translation to SMT for the argument below.
   Let's call their conjunction $Ax$.
   Assume the hypotheses.
   \\
   \begin{enumerate}
   \item Suppose, for a contradiction, that $\Gtcp$ is not a theorem of ACL2.
      Hence neither is the implication $(Ax \rightarrow \Gtcp)$.
   \item Then by Goedel Completeness, there is a model $M_{acl2}$ satisfying
      $Ax$ and $(\neg \Gtcp)$.
   \item We can then construct a model $M_{smt}$ for SMT of the translation
      $(\neg \Gsmt)$ of $(\neg \Gtcp)$ that satisfies the SMT axioms. \Yan{This
        is a question for Matt. What does ``can then construct a model'' mean?
        Does it mean we can safely claim there will be a model or do we have to
        construct it?}
   \item By a suitable completion/closure procedure, we can build a model
      $M_{smt'}$ for SMT in which $M_{smt}$ sits as a submodel -- so $M_{smt'}$
      also satisfies $(\neg \Gsmt)$, since $\Gsmt$ is quantifier-free. \Yan{This
      is a question for Matt. For this proof step, what does it mean by ``a
      suitable completion/closure procedure''? Is there an example that I can
      follow? What sorts (except int to real and cons of lists) needs to be
      included in this completion/closure procedure?}
   \item Therefore $\Gsmt$ is not a first-order theorem of SMT (else it would
      hold in $M_{smt'}$).
   \item By Assumption~\ref{sec:soundness:assumpt2}, $\Gsmt$ is not verifiable
     by the intended SMT solvers, contradicting the hypotheses.
   \end{enumerate}

\section{SMT Sorts and the Axioms}\label{sec:soundness:sorts}
The logic of ACL2 is more expressive than that of the SMT solvers. An ACL2 type
recognizer could be an arbitrary ACL2 function. Therefore it
is more straightforward to represent SMT sorts and their axioms in ACL2.

We identify a set of SMT sorts and functions that are supported in Smtlink.
Smtlink embeds the SMT sorts and functions in ACL2. The trusted clause-processor
directly translates from the SMT sorts embedded in ACL2 to SMT.
For each SMT sort, we identify the set of SMT axioms that corresponds to the
sorts. The axioms must be established in ACL2 to ensure soundness of the
translation.

\subsection{BoolSort}

\subsection{IntSort}
\subsection{RealSort}
ACL2 doesn't really have real numbers, though it has the rationale to support
rational numbers. The rational numbers form an ordered field. The Artin-Schreier
theorem shows that the algebraic closure of the rationals is a real closed
field\footnote{A real-closed field is a field that has the same first-order
  properties as the field of the real numbers.}. This means that any model of
the rationals corresponds to a model in the algebraic extension, which is also a
model in the real numbers. \Yan{To Mark: I figured this description might be too
  simple but I find it very hard to formulate.} Here, we need to establish that
the rational numbers of ACL2 forms an ordered field.

% We identify a set of functions over the rational numbers:
% \texttt{rational-binary-+}, \texttt{rational-binary-*},
% \texttt{rational-unary--}, \texttt{rational-unary-/}, \texttt{rational-<} and
% \texttt{rational-=}. We then identify the set of axioms over these functions
% that formulates an ordered field. The set of axioms also include type axioms.

% \Yan{Could we work out what the axioms are together?}
% http://homepages.math.uic.edu/~kauffman/axioms1.pdf

% \Yan{Mention ACL2(r) here.}

% \subsubsection{Int to Real}
% We convert integers to rationals through the function \texttt{to-rat}. The
% axioms for this function are:

% \begin{itemize}
%   \item \texttt{to-rat} takes an integer and returns a rational
%   \item \texttt{to-rat} is an identity
% \end{itemize}

\subsection{Datatypes}

\subsection{Arrays}

\subsection{Uninterpreted Sorts}

\subsection{Uninterpreted Functions}


\section{Missing pieces}

\subsection{ACL2 Symbols?}

