%% The following is a directive for TeXShop to indicate the main file
%%!TEX root = diss.tex

\chapter{Abstract}

SMT solvers are automated decision procedures for domain-specific solving
problems including but not limited to boolean, integer, and real theories,
uninterpreted function theories, array theories, and bit-vector theories. They
have become a main method widely applied in the industry for solving software
and hardware formal verification problems. Though theorem provers are largely
interactive, they are often resorted to for its powerful induction capability,
programmability and proof management. Any moderate-complexity problem will
require both methods but switching between tools requires modelling the same
system in each tool. This introduces huge verification overhead and could be
erroneous.

Existing work on the integration of SMT solvers and theorem provers focuses on
proof reconstruction for soundness, but lacks emphasis on the usability and
proof efficiency. To use such integration, the theorem prover term needs to be
in a form that is ready for direct translation. Since proof reconstruction is
undecidable, it is reported that it could be time consuming and even unsolvable.

In this thesis, we propose using reflection for integrating SMT solvers into
theorem provers. We demonstrate our method using the ACL2 theorem prover and the
Z3 SMT solver. To bridge the gap between the untyped first-order logic of ACL2
and the many-sorted logic of SMT solvers, we apply reflection over the ACL2
terms to achieve adding user hypotheses, function expansion, type inference,
term replacement and many other transformations. By using reflection, the user
is freed from manually transforming theorem prover terms into a shape that is
directly translatable. In addition, soundness of these transformations is
easily achieved and does not require extra proof time, therefore promotes proof
efficiency. The framework is also built to be extensible, allowing new
transformations to be integrated. We analyze our tool by conducting formal
verification of asynchronous circuits and machine learning convex optimization
problems.

% Consider placing version information if you circulate multiple drafts
%\vfill
%\begin{center}
%\begin{sf}
%\fbox{Revision: \today}
%\end{sf}
%\end{center}
